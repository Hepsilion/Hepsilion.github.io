\documentclass[letterpaper, UTF8, 11pt]{article}
\usepackage{hyperref}
\usepackage{geometry}
\usepackage[T1]{fontenc}
\usepackage[sc,osf]{mathpazo}
\usepackage{sectsty}
\usepackage{indentfirst}
\usepackage{color}
\usepackage{ctex}
\usepackage{fontawesome}
\usepackage{setspace}

\def\name{\textbf{\textcolor[rgb]{0.00, 0.00, 0.00}{\fontsize{30pt}{30pt}吴庭明}} ~~~~~~~~~ \fontsize{15pt}{15pt}}

% The following metadata will show up in the PDF properties
\hypersetup{
	colorlinks = true,
	urlcolor = black,
	pdfauthor = {\name},
	pdfkeywords = {economics, statistics, mathematics},
	pdftitle = {\name: Curriculum Vitae},
	pdfsubject = {Curriculum Vitae},
	pdfpagemode = UseNone
}

\geometry{
	body={7.0in, 8.5in},
	top=0.5in,   %1.5
	left=0.55in, %1.0
	right=0.55in,
	bottom=0.5in
}

% Customize page headers
\pagestyle{myheadings}
\markright{\name}
\thispagestyle{empty}

% Custom section fonts
\usepackage{sectsty}
\sectionfont{\rmfamily\mdseries\Large}
\subsectionfont{\rmfamily\mdseries\itshape\large}

% Other possible font commands include:
% \ttfamily for teletype,
% \sffamily for sans serif,
% \bfseries for bold,
% \scshape for small caps,
% \normalsize, \large, \Large, \LARGE sizes.

% Don't indent paragraphs.
\setlength\parindent{0em}

% Make lists without bullets
%\renewenvironment{itemize}{
%	\begin{list}{}{
%			\setlength{\leftmargin}{0pt}
%			\setlength{\itemsep}{0pt}
%			\setlength{\parsep}{0pt}
%			\setlength{\parskip}{0pt}
%		}
%	}{
%\end{list}
%}

%set margin between item
\usepackage{enumitem}
\setenumerate[1]{itemsep=0pt,partopsep=0pt,parsep=\parskip,topsep=5pt}
\setitemize[1]{itemsep=0pt,partopsep=0pt,parsep=\parskip,topsep=5pt}
\setdescription{itemsep=0pt,partopsep=0pt,parsep=\parskip,topsep=5pt}

\begin{document}
	
	% Place name at left
	\noindent{\bf \name} 
	\vspace{0.1in}

	\begin{minipage}{0.45\linewidth}
		\begin{tabular}{ll}
			电话:   & (+86) 15921576683 \\
			微信:   & 2428091608 \\
		\end{tabular}
	\end{minipage}
	\begin{minipage}{0.45\linewidth}
		\begin{tabular}{ll}
			邮箱:   & \href{mailto:wutingming@hotmail.com}{ wutingming@hotmail.com} \\
			Github: & \href{https://github.com/Hepsilion}{https://github.com/Hepsilion}\\
		\end{tabular}
	\end{minipage}
	\vspace{-0.1in}
	
	\section*{\textbf{求职意向}}\vspace{-0.15in}
		\textbf{Java研发工程师}
	\vspace{-0.25in}
	
	\section*{\textbf{专业技能}}\vspace{-0.15in}
	\begin{itemize}
		%\item 良好的英语读写能力 (英语6级:508分)
		\item Java、Android、SpringMVC、Python、HTML/CSS/JavaScript、C/C++、Linux
	\end{itemize}
	\vspace{-0.32in}
	
	\section*{\textbf{教育经历}}\vspace{-0.15in}
	\begin{itemize}
		\item \textbf{硕士},2015/09 - 2018/07 ~~~~~~~~华东师范大学,软件工程  ~~~~~~~~~排名:15/70
		\item \textbf{本科},2011/09 - 2015/07 ~~~~~~~~华东师范大学,软件工程  ~~~~~~~~~绩点:3.43/4.0
	\end{itemize}
	\vspace{-0.32in}
	
	\section*{\textbf{荣誉}}\vspace{-0.15in}
	\begin{itemize}
		\item \textbf{2017年全国大学生物联网设计竞赛全国总决赛一等奖}~~~~~~~~~~~~~~~~~~~~~~~~~~~~~~~~~~~~~~~~~~~~~~~\textbf{2017/09}
		\item \textbf{第三届全国高校云计算应用创新大赛华东赛区总决赛三等奖}~~~~~~~~~~~~~~~~~~~~~~~~~~~~~~~~~~~~~~~~\textbf{2017/03}
		\item \textbf{2014年华东师范大学三等奖学金}~~~~~~~~~~~~~~~~~~~~~~~~~~~~~~~~~~~~~~~~~~~~~~~~~~~~~~~~~~~~~~~~~~~~~~~~~~\textbf{2014/11}
	\end{itemize}
	\vspace{-0.32in}
	
	\section*{\textbf{工作经历}}\vspace{-0.15in}
	\begin{itemize}
		\item \textbf{Intel}, \emph{实习,WebRTC开发组,软件开发与测试}~~~~~~~~~~~~~~~~~~~~~~~~~~~~~~~~~~~~~~~~~~~~~~~~\textbf{2017/07 $\sim$ 至今}
		
		1. 基于WebRTC开发端到端的实时通信解决方案,为用户提供P2P和Conference两种音视频服务模式\\
		2. 参与Android版本的客户端SDK的部分特征开发实现、bug修复\\
		3. 编写Python脚本,实现WebRTC源码、SDK源码的自动化编译、打包、发布与下载
		\item \textbf{惠普}, \emph{实习,软件开发与服务部门,软件开发}~~~~~~~~~~~~~~~~~~~~~~~~~~~~~~~~~~~~~~~~~~~~~~~~~~~\textbf{2014/08 $\sim$ 2014/11}
		
		1. 参与公司项目管理工具的开发,使用SpringMVC实现客户端和服务端部分业务逻辑。
	\end{itemize}
	\vspace{-0.32in}
	
	\section*{\textbf{项目经历}}\vspace{-0.15in}
	\begin{itemize}	
		\item \textbf{虚拟现实自行车:}{使用Unity3D、Android和Arduino实现的虚拟自行车骑行系统。}~\textbf{2016/12 $\sim$ 2017/01}
		
		1. 主要负责项目的\textbf{方案设计、通信协议制定、客户端及其与服务端通讯过程的实现};\\
		2. 使用Socket实现\textbf{Android}客户端与Unity服务程序之间的VR数据交互,能够支持多人在线的场景。
		%(1) Arduino单片机将真实自行车的实时把手转角和速度发送给Unity服务程序,Unity服务程序根据转角和速度控制虚拟场景中自行车的运动,并渲染出对应的场景。\\
		%(2) Android客户端将手机实时方向发送给Unity服务程序,Unity服务程序根据方向数据返回对应视角的图像,Android客户端对图像流进行显示。\\
		%(3) 自行车的转角检测:将滑动变阻器固定在自行车龙头的转动轴上,把手转动带动滑动变阻器滑片的移动(电阻也发生变化),Arduino单片机实时检测滑动变阻器的电压,并根据电压值和转角值之间的线性关系计算出把手转角。\\
		%(4) 自行车的速度检测:在自行车飞轮上吸附一颗磁铁,将通电的接近开关固定在车架上(磁铁可以靠近的地方);当磁铁靠近接近开关时,接近开关闭合并输出脉冲。将脉冲从无到有看作一次上升沿,一次上升沿的到来触发中断,执行中断处理程序,中断处理程序通过统计相邻两次上升沿到来的时间间隔计算出车轮的速度。\\
		%(5) 遇到的问题:\\
		% 1. 转角检测过程中电压不稳定(电压值发生微小变化)带来的抖动:在每一次虚拟自行车转角更新时,记录新转角对应的电压值(简称为记录值)。如果当前测得的电压值与记录值差距小于很小的固定阈值,则忽略此次电压变化;否则计算新转角并更新记录值为当前电压值。\\
		% 2. 速度检测过程中优化Arduino单片机发送的数据量:类似于1的解决方法,当相邻两次测得的速度值变化小于很小的固定阈值时,忽略此次速度变化;仅在速度变化较大时,才向Unity服务程序发送真实自行车的速度。
		% 3. 图像数据延迟、数据压缩问题。
		\vspace{0.03in}
		
		%\item \textbf{\href{https://github.com/Hepsilion/SecurityGuard}{手机安全卫士}}~~~~~~~~~~~~~~~~~~~~~~~~~~~~~~~~~~~~~~~~~~~~~~~~~~~~~~~~~~~~~~~~~~~~~~~~~~~~~~~~~~~~\textbf{2016/10 $\sim$ 2016/12}
		
		%1. 本项目是一个手机安全管理App,用于管理手机中的数据以及保护用户隐私。\\
		%2. 作为项目参与者,\textbf{实现黑名单电话拦截、病毒查杀、缓存清理、流量统计等模块}。\\
		%3. 使用\textbf{AIDL接口}调用Android底层方法实现上述功能
		%\vspace{0.01in}
		
		\item \textbf{\href{https://github.com/Hepsilion/MapProject}{位置共享应用:}}{提供位置共享服务的App,类似于微信中的位置共享。}~~~~~~~~~~~~~~~~~\textbf{2013/02 $\sim$ 2013/06}
		
		1. 主要负责项目的\textbf{方案设计,客户端与服务器程序的实现};\\
		2. 利用\textbf{多线程}实现多并发场景,每个用户通过\textbf{Socket}与服务器程序进行通信;\\
		3. 客户端利用\textbf{Service}运行在后台实时获取用户的位置数据,通过服务器和其他客户端之间接共享位置。
		\vspace{0.03in}	
		
		\item \textbf{\href{https://github.com/Hepsilion/cloudsim}{CloudSim:}{云计算资源调度仿真工具}} ~~~~~~~~~~~~~~~~~~~~~~~~~~~~~~~~~~~~~~~~~~~~~~~~~~~~~~~~~ \textbf{2015/09 $\sim$ 2017/05}
		
		1. 参与项目的开发、修复部分bug、实现涉及软错误节能调度的主要功能;\\
		2. 利用CloudSim仿真云计算集群做一些资源节能调度方面的科研,完成论文、专利、软件著作权若干篇。
		\vspace{0.03in}	
	\end{itemize}
	\vspace{-0.32in}
	
	\section*{\textbf{科研经历}}\vspace{-0.15in}
	\begin{itemize}	
		\item \small 论文: (在投)\\
		1. Energy-Aware Virtual Machine Allocation in the Cloud with Resource Reservation: An Evolutionary Approach\\
		2. Energy-Efficient Task Scheduling for Workflow Applications in DVFS-Enabled Cloud Considering Soft Errors
		\item 专利:一个面向节能的启发式云计算资源分配与调度方法 (专利号:201610966411.5)
		\item 软件著作权:一种面向节能的云任务调度软件 (受理号:2017R11S093585)
	\end{itemize}
	\vspace{-0.3in}
	
%	\begin{center}
%		\begin{footnotesize}
%			%最后更新: \today 
%		\end{footnotesize}
%	\end{center}
\end{document}