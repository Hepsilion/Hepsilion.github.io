\documentclass[letterpaper, UTF8, 11pt]{article}
\usepackage{hyperref}
\usepackage{geometry}
\usepackage[T1]{fontenc}
\usepackage[sc,osf]{mathpazo}
\usepackage{sectsty}
\usepackage{indentfirst}
\usepackage{color}
\usepackage{ctex}
\usepackage{fontawesome}
\usepackage{setspace}

\def\name{\textbf{\textcolor[rgb]{0.00, 0.00, 0.00}{\fontsize{30pt}{30pt}吴庭明}} ~~~~~~~~~ \fontsize{15pt}{15pt}}

% The following metadata will show up in the PDF properties
\hypersetup{
	colorlinks = true,
	urlcolor = black,
	pdfauthor = {\name},
	pdfkeywords = {economics, statistics, mathematics},
	pdftitle = {\name: Curriculum Vitae},
	pdfsubject = {Curriculum Vitae},
	pdfpagemode = UseNone
}

\geometry{
	body={7.0in, 8.5in},
	top=0.5in,   %1.5
	left=0.4in, %1.0
	right=0.5in,
	bottom=0.5in
}

% Customize page headers
\pagestyle{myheadings}
\markright{\name}
\thispagestyle{empty}

% Custom section fonts
\usepackage{sectsty}
\sectionfont{\rmfamily\mdseries\Large}
\subsectionfont{\rmfamily\mdseries\itshape\large}

% Other possible font commands include:
% \ttfamily for teletype,
% \sffamily for sans serif,
% \bfseries for bold,
% \scshape for small caps,
% \normalsize, \large, \Large, \LARGE sizes.

% Don't indent paragraphs.
\setlength\parindent{0em}

% Make lists without bullets
%\renewenvironment{itemize}{
%	\begin{list}{}{
%			\setlength{\leftmargin}{0pt}
%			\setlength{\itemsep}{0pt}
%			\setlength{\parsep}{0pt}
%			\setlength{\parskip}{0pt}
%		}
%	}{
%\end{list}
%}

%set margin between item
\usepackage{enumitem}
\setenumerate[1]{itemsep=0pt,partopsep=0pt,parsep=\parskip,topsep=5pt}
\setitemize[1]{itemsep=0pt,partopsep=0pt,parsep=\parskip,topsep=5pt}
\setdescription{itemsep=0pt,partopsep=0pt,parsep=\parskip,topsep=5pt}

\begin{document}
	
	% Place name at left
	\noindent{\bf \name} 
	\vspace{0.1in}

	\begin{minipage}{0.45\linewidth}
		\begin{tabular}{ll}
			电话:   & (+86) 15921576683 \\
			微信:   & Hepsilion \\
		\end{tabular}
	\end{minipage}
	\begin{minipage}{0.45\linewidth}
		\begin{tabular}{ll}
			邮箱:   & \href{mailto:wutingming@hotmail.com}{ wutingming@hotmail.com} \\
			Github: & \href{https://github.com/Hepsilion}{https://github.com/Hepsilion}\\
		\end{tabular}
	\end{minipage}
	\vspace{-0.1in}
	
	\section*{\textbf{求职意向}}\vspace{-0.12in}
	\begin{itemize}
		\item \textbf{Java研发工程师}
	\end{itemize}
	\vspace{-0.25in}
	
	\section*{\textbf{专业技能}}\vspace{-0.12in}
	\begin{itemize}
		\item 熟悉JVM、RPC、NIO、并发
		\item 熟悉常见的数据结构与算法、设计模式
		\item 熟悉JUC、Netty、Tomcat、Spring、SpringCloudGateway等开源框架与工具的源码
		\item 了解Kafka、MySQL、Redis、日志框架、计算机网络

	\end{itemize}
	\vspace{-0.32in}
	
	\section*{\textbf{教育经历}}\vspace{-0.12in}
	\begin{itemize}
		\item \textbf{硕士}: 华东师范大学,计算机科学与软件工程学院,软件工程专业~~~~~~~~~~~~~~~~~~~~~~\textbf{2015/09 $\sim$ 2018/07}
		\item \textbf{本科}: 华东师范大学,软件学院,软件工程(嵌入式系统)专业~~~~~~~~~~~~~~~~~~~~~\textbf{2011/09 $\sim$ 2015/07}
	\end{itemize}
	\vspace{-0.32in}
	
	\section*{\textbf{工作经历}}\vspace{-0.12in}
	\begin{itemize}
		\item \textbf{百度}: \emph{高级研发工程师,智能云物联网部、地图开放平台业务部}~~~~~~~~~~~~~~~~~~~~~~~~~~\textbf{2018/07 $\sim$ 至今}
		\item \textbf{英特尔}: \emph{实习研发工程师,软件开发与服务部}~~~~~~~~~~~~~~~~~~~~~~~~~~~~~~~~~~~~~~~~~~~~~~~~~~\textbf{2017/07 $\sim$ 2017/12}
		\item \textbf{惠普}: \emph{实习研发工程师,软件开发与服务部}~~~~~~~~~~~~~~~~~~~~~~~~~~~~~~~~~~~~~~~~~~~~~~~~~~~~~\textbf{2014/08 $\sim$ 2014/11}
	\end{itemize}
	\vspace{-0.32in}
	
	\section*{\textbf{项目经历}}\vspace{-0.12in}
	\begin{itemize}	
		\item \textbf{\href{https://lbsyun.baidu.com/solutions/logisticsmap}{智能物流系统}:}{围绕物流场景的路线规划、分区分单与运输管理等系统}~~~~~~~~~~~~~~~~~\textbf{2020/02 $\sim$ 至今}
		
		1. 参与百度区划平台核心功能的设计与研发,持续迭代优化,\\
		2. \\
		3. 参与用户需求对接与功能设计,运维物流货运路线规划与批量算路API \\
		4. 设计并开发火车、飞机、货车与步行的联运路线规划系统,为用户提供多交通方式的路线规划功能;\\
		5. 设计各服务的稳定性优化方案,推动组内同事对服务监控进行梳理与完善、对服务可用性进行定义并可视化,制定标准规范可用性值班同事的职责,有效建立了完善的运维监控方案。
		
		\vspace{0.03in}

		\item \textbf{\href{https://lbsyun.baidu.com/solutions/scheduling}{物流智能调度}:}{使用调度算法为物流企业提供运输调度管理的智能化平台。}~~~~~~~~~~~\textbf{2019/07 $\sim$ 2020/01}
		
		1.开发参数管理平台,设计参数集、参数模板,为不同场景和不同用户提供高效的参数配置与查询能力;\\
		2.设计路线规划策略平台,屏蔽底层不同地图的差异,支持分用户分功能的策略配置与私有路线规划能力;\\
		3.设计算法调试平台并完成功能迭代,支持在本地使用线上资源对算法调试分析,大幅提高算法调试效率;
		
		\vspace{0.03in}

		\item \textbf{微服务网关:}{为组内产品提供服务注册与发现、流量分发等能力的微服务化组件。}~~\textbf{2019/04 $\sim$ 2019/06}
		
		1.基于SpringCloud开发网关,设计数据结构与算法对用户请求进行寻址,大幅提升路由的效率与准确性;\\
		2.基于ETCD开发配置中心,支持业务端上报路由、配置和健康状态等数据,并向网关提供服务发现能力;\\
		3.设计微服务SDK,业务端集成后自动完成路由收集与上报、配置监听与更新,降低业务接入的复杂程度。
		
		\vspace{0.03in}
		
		\item \textbf{\href{https://cloud.baidu.com/product/dugo.html}{物联网车辆云}:}{为车辆提供数据接入、存储、分析与控制的车联网云服务平台。}~~~~~\textbf{2018/07 $\sim$ 2019/03}
		
		1.设计开发车联网设备注册服务,支持用户向平台注册HTTP、GB/T32960、MQTT等协议的车载设备;\\
		2.基于Netty开发GB/T32960、JT808协议的车载设备数据接入系统,支持数据上报与指令控制等功能;\\
		3.对接客户需求,设计满足用户场景的自定义协议、并完成系统开发,支持客户上千辆电动车接入与运营。
		
		\vspace{0.03in}
		
	\end{itemize}
	\vspace{-0.32in}
	
	\section*{\textbf{自我评价}}\vspace{-0.12in}
	\begin{itemize}	
		\item 具备良好的团队精神与责任感;
		% 擅长简化代码逻辑,
		具备强烈的工程质量意识;自驱力强,对技术和解决问题有持续的热情。
	\end{itemize}
	\vspace{-0.32in}
\end{document}
