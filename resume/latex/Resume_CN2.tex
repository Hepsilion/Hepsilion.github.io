\documentclass[letterpaper, UTF8, 11pt]{article}
\usepackage{hyperref}
\usepackage{geometry}
\usepackage[T1]{fontenc}
\usepackage[sc,osf]{mathpazo}
\usepackage{sectsty}
\usepackage{indentfirst}
\usepackage{color}
\usepackage{ctex}
\usepackage{fontawesome}
\usepackage{setspace}

\def\name{\textbf{\textcolor[rgb]{0.00, 0.00, 0.00}{\fontsize{30pt}{30pt}吴庭明}} ~~~~~~~~~ \fontsize{15pt}{15pt}}

% The following metadata will show up in the PDF properties
\hypersetup{
	colorlinks = true,
	urlcolor = black,
	pdfauthor = {\name},
	pdfkeywords = {economics, statistics, mathematics},
	pdftitle = {\name: Curriculum Vitae},
	pdfsubject = {Curriculum Vitae},
	pdfpagemode = UseNone
}

\geometry{
	body={7.0in, 8.5in},
	top=0.5in,   %1.5
	left=0.65in, %1.0
	right=0.65in,
	bottom=0.5in
}

% Customize page headers
\pagestyle{myheadings}
\markright{\name}
\thispagestyle{empty}

% Custom section fonts
\usepackage{sectsty}
\sectionfont{\rmfamily\mdseries\Large}
\subsectionfont{\rmfamily\mdseries\itshape\large}

% Other possible font commands include:
% \ttfamily for teletype,
% \sffamily for sans serif,
% \bfseries for bold,
% \scshape for small caps,
% \normalsize, \large, \Large, \LARGE sizes.

% Don't indent paragraphs.
\setlength\parindent{0em}

% Make lists without bullets
%\renewenvironment{itemize}{
%	\begin{list}{}{
%			\setlength{\leftmargin}{0pt}
%			\setlength{\itemsep}{0pt}
%			\setlength{\parsep}{0pt}
%			\setlength{\parskip}{0pt}
%		}
%	}{
%\end{list}
%}

%set margin between item
\usepackage{enumitem}
\setenumerate[1]{itemsep=0pt,partopsep=0pt,parsep=\parskip,topsep=5pt}
\setitemize[1]{itemsep=0pt,partopsep=0pt,parsep=\parskip,topsep=5pt}
\setdescription{itemsep=0pt,partopsep=0pt,parsep=\parskip,topsep=5pt}

\begin{document}
	
	% Place name at left
	\noindent{\bf \name} 
	\vspace{0.1in}

	\begin{minipage}{0.45\linewidth}
		\begin{tabular}{ll}
			电话:   & (+86) 15921576683 \\
			微信:   & Hepsilion \\
		\end{tabular}
	\end{minipage}
	\begin{minipage}{0.45\linewidth}
		\begin{tabular}{ll}
			邮箱:   & \href{mailto:wutingming@hotmail.com}{ wutingming@hotmail.com} \\
			Github: & \href{https://github.com/Hepsilion}{https://github.com/Hepsilion}\\
		\end{tabular}
	\end{minipage}
	\vspace{-0.1in}
	
	\section*{\textbf{求职意向}}\vspace{-0.12in}
	\begin{itemize}
		\item \textbf{Java研发工程师}
	\end{itemize}
	\vspace{-0.25in}
	
	\section*{\textbf{专业技能}}\vspace{-0.12in}
	\begin{itemize}
		\item 熟练使用Java,对Java并发、JVM有较深入的理解
		\item 翻阅过Spring、Netty、Tomcat等开源框架与工具的源码
	\end{itemize}
	\vspace{-0.32in}
	
	\section*{\textbf{教育经历}}\vspace{-0.12in}
	\begin{itemize}
		\item \textbf{硕士},华东师范大学,计算机科学与软件工程学院~~~软件工程~~~~~~~~~~~~~~~~~~~~~~~~~~\textbf{2015/09 $\sim$ 2018/07}
		\item \textbf{本科},华东师范大学,软件学院~~~~~~~~~~~~~~~~~~~~~~~~~~~~软件工程(嵌入式系统)~~~~~~~\textbf{2011/09 $\sim$ 2015/07}
	\end{itemize}
	\vspace{-0.32in}
	
	\section*{\textbf{荣誉}}\vspace{-0.12in}
	\begin{itemize}
		\item \textbf{2018年度上海市优秀毕业研究生}~~~~~~~~~~~~~~~~~~~~~~~~~~~~~~~~~~~~~~~~~~~~~~~~~~~~~~~~~~~~~~~~~~~~~\textbf{2018/06}
	\end{itemize}
	\vspace{-0.32in}
	
	\section*{\textbf{工作经历}}\vspace{-0.12in}
	\begin{itemize}
		\item \textbf{百度}, \emph{研发工程师,智能云物联网部、地图开放平台业务部}~~~~~~~~~~~~~~~~~~~~~~~~~~~~\textbf{2018/07 $\sim$ 至今}
		\item \textbf{英特尔}, \emph{实习研发工程师,软件开发与服务部}~~~~~~~~~~~~~~~~~~~~~~~~~~~~~~~~~~~~~~~~~~~~~~~\textbf{2017/07 $\sim$ 2017/12}
		\item \textbf{惠普}, \emph{实习研发工程师,软件开发与服务部}~~~~~~~~~~~~~~~~~~~~~~~~~~~~~~~~~~~~~~~~~~~~~~~~~~\textbf{2014/08 $\sim$ 2014/11}
	\end{itemize}
	\vspace{-0.32in}
	
	\section*{\textbf{项目经历}}\vspace{-0.12in}
	\begin{itemize}	
		\item \textbf{\href{https://lbsyun.baidu.com/solutions/scheduling}{智能物流系统}:}{围绕物流场景的路线规划、分区分单与运输管理等系统}~~~~~~~~~~~~~~~\textbf{2020/02 $\sim$ 至今}
		
		1. 物流货运路线规划与批量算路API\\
		2. \\
		3. 
		
		\vspace{0.03in}

		\item \textbf{\href{https://lbsyun.baidu.com/solutions/scheduling}{物流智能调度}:}{使用调度算法为物流企业提供运输调度管理的智能化平台。}~~~~~~~~~\textbf{2019/07 $\sim$ 2020/01}
		
		1. 负责项目的\textbf{方案设计、通信协议制定、客户端及其与服务端通讯过程的实现};\\
		2. 使用Socket实现\textbf{Android}客户端与Unity服务程序之间的VR数据交互,支持多人在线场景。
		
		\vspace{0.03in}

		\item \textbf{微服务网关:}{为组内产品提供服务注册与发现和流量分发等能力的微服务化组件。}~\textbf{2019/04 $\sim$ 2019/06}
		
		1. 基于Spring Cloud Gateway定制开发了网关,设计数据结构与算法对路由逻辑进行优化\\
		2. 基于ETCD设计开发了服务注册与发现组件\\
		3. 设计并开发了微服务客户端,简化各业务接入过程
		
		\vspace{0.03in}
		
		\item \textbf{\href{https://cloud.baidu.com/product/dugo.html}{物联网车辆云}:}{为车辆提供数据接入、存储、分析与控制的车联网云服务平台。}~~~~\textbf{2018/07 $\sim$ 2019/03}
		
		1. 设计并开发车联网设备接入系统\\
		2. 基于Netty设计并开发了符合GB/T 32960协议的车载设备数据接入系统,支持数据上报与指令控制等功能
		
		\vspace{0.03in}
		
	\end{itemize}
	\vspace{-0.32in}
\end{document}
