\documentclass[letterpaper, UTF8]{article}
\usepackage{hyperref}
\usepackage{geometry}
\usepackage[T1]{fontenc}
\usepackage[sc,osf]{mathpazo}
\usepackage{sectsty}


\def\name{Tingming Wu}

% The following metadata will show up in the PDF properties
\hypersetup{
	colorlinks = true,
	urlcolor = black,
	pdfauthor = {\name},
	pdfkeywords = {economics, statistics, mathematics},
	pdftitle = {\name: Curriculum Vitae},
	pdfsubject = {Curriculum Vitae},
	pdfpagemode = UseNone
}

\geometry{
	body={7.0in, 8.5in},
	top=0.5in,   %1.5
	left=0.55in, %1.0
	right=0.55in,
	bottom=0.5in
}

% Customize page headers
\pagestyle{myheadings}
\markright{\name}
\thispagestyle{empty}

% Custom section fonts
\usepackage{sectsty}
\sectionfont{\rmfamily\mdseries\Large}
\subsectionfont{\rmfamily\mdseries\itshape\large}

% Other possible font commands include:
% \ttfamily for teletype,
% \sffamily for sans serif,
% \bfseries for bold,
% \scshape for small caps,
% \normalsize, \large, \Large, \LARGE sizes.

% Don't indent paragraphs.
\setlength\parindent{0em}

% Make lists without bullets
%\renewenvironment{itemize}{
%	\begin{list}{}{
%			\setlength{\leftmargin}{1.5em}
%			\setlength{\itemsep}{0pt}
%			%\setlength{\parsep}{0pt}
%			%\setlength{\parskip}{0pt}
%		}
%	}{
%\end{list}
%}

%set margin between item
%\usepackage{enumitem}
%\setenumerate[1]{itemsep=0pt,partopsep=0pt,parsep=\parskip,topsep=5pt}
%\setitemize[1]{itemsep=0pt,partopsep=0pt,parsep=\parskip,topsep=5pt}
%\setdescription{itemsep=0pt,partopsep=0pt,parsep=\parskip,topsep=5pt}

\begin{document}
	
	% Place name at left
	{\huge \name}
	
	% Alternatively, print name centered and bold:
	%\centerline{\huge \bf \name}
	
	\vspace{0.15in}  % add by Kevin Wu
	
	\begin{minipage}{0.45\linewidth}
		\href{http://www.ecnu.edu.cn/}{East China Normal University} \\
		\href{http://www.sei.ecnu.edu.cn}{Department of Computer Science}  \\
		Putuo, Shanghai 200062
	\end{minipage}
	\begin{minipage}{0.45\linewidth}
		\begin{tabular}{ll}
			Phone: & (+86) 15921576683 \\
			Email: & \href{mailto:wutingming@hotmail.com}{\tt wutingming@hotmail.com} \\
			Github: &  \href{https://github.com/Hepsilion}{https://github.com/Hepsilion}\\
		\end{tabular}
	\end{minipage}
	\vspace{-0.1in}
	
	\section*{\textbf{Job Objective}}\vspace{-0.05in}
	\begin{itemize}
		\item Java Development Engineer
	\end{itemize}
	\vspace{-0.25in}
	
	\section*{\textbf{Skills}}\vspace{-0.05in}
	\begin{itemize}
		\item Java, Android, SpringMVC, Python, HTML/CSS/JavaScript, C/C++, Linux
	\end{itemize}
	\vspace{-0.32in}
	
	\section*{\textbf{Education}}\vspace{-0.05in}
	\begin{itemize}
		\item M.S. (Software Engineering), 2015-2018, East China Normal University.
		\item B.S. (Software Engineering), 2011-2015, East China Normal University.
	\end{itemize}	
	\vspace{-0.32in}
	
	\section*{\textbf{Honors}}\vspace{-0.05in}
	\begin{itemize}
		\item First prize in the Final Contest of TIIC National Undergraduate IOT Design Contest, 2017
		\item Third prize in East China Area Contest of National Undergraduate Cloud Computing Application Contest
		\item Third-class scholarship from East China Normal University
	\end{itemize}
	\vspace{-0.32in}
	
	\section*{\textbf{Working Experience}}\vspace{-0.05in}
	\begin{itemize}
		\item \textbf{Intel, Shanghai}, \emph{Intern in Intel WebRTC development team}, 07/2017 - present  \\
		 1. We developed a WebRTC-based realtime communication solution to provide P2P and Conference services.\\
		 2. I wrote Python scripts for building WebRTC source code and for compiling, packing, publishing and downloading SDK automatically.\\
		 3. I engaged in implementing part of SDK features and repairing bugs existing in SDK souce code
		\item \textbf{Hewlett-Packard, Shanghai}, \emph{Intern in Software development team}, 08/2014 - 11/2014  \\
		 1. I enaged in developing an internal project management tool and implementing part of business logic using SpringMVC.
	\end{itemize}
	\vspace{-0.32in}
	
	\section*{\textbf{Projects}}
	\begin{itemize}
		\item \textbf{Virtual Reality Bike}. 12/2016 - 01/2017\\
		1. It is a virtual reality bike system implemented using Unity3D, Android and Arduino.\\
		2. I was responsible for project design, communication protocol design and implementing communication process between clients and server.\\
		3. VR data communication between Android Client and Unity Server is implemented using Socket and our project supports multiple clients sharing the same scene.
		
		\item \textbf{Location Sharing Map}. 02/2013 - 06/2013\\
		1. Its is An app providing location sharing like the function in WeChat.\\
		1. I was responsible for project design and implement client program and server program.\\
		2. Concurrent scene is implemented using thread and each communicate with server by a socket.\\
		3. A client app fetches its location data using a Service and shares it with peer client through the server. 
		
		\item \textbf{Cloudsim}. 09/2015 - 01/2016\\
		1. It is a simulation tool for cloud computing scheduling study.\\		
		1. I engaged in project development, repair some bugs and implement some functions.\\
		2. I used CloudSim to do some research and complete some papers, patents and software copyrights.		
		
	\end{itemize}
	\vspace{-0.32in}
	
	\section*{\textbf{Research Experience}}\vspace{-0.15in}
	\begin{itemize}	
		\item \small Papers: (under review)\\
		1. Energy-Aware Virtual Machine Allocation in the Cloud with Resource Reservation: An Evolutionary Approach.\\
		2. Energy-Efficient Task Scheduling for Workflow Applications in DVFS-Enabled Cloud Considering Soft Errors.
		\item Patent: A heuristic method for energy-efficient cloud resource scheduling and allocation (Patent number:201610966411.5)
		\item Software Copyright: A software for energy-efficient cloud task scheduling (Acceptance number: 2017R11S093585)
	\end{itemize}
	\vspace{-0.32in}
	
	\bigskip
	% Footer
	%\begin{center}
	%	\begin{footnotesize}
	%		Last updated: \today 
	%	\end{footnotesize}
	%\end{center}
\end{document}